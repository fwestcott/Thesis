\chapter{\label{ch:3-Model Development}Model Development}

\minitoc

 \section{Introduction}

 \section{Methods}

 \section{Results}

\subsection{Effect of media FA concentration on TG accumulation and autophagy}

By culturing Huh7 cells in increasing concentrations of OPLA FA mix, intracellular TG content significantly increased (Figure \ref{fig:LFHF TG}) but despite the substantial increase in TG accumulation, there was no effect on autophagic gene expression (Figure \ref{fig:LFHF ATG genes}).

\begin{figure}[h!]
  \centering
  {\phantomsubcaption\label{fig:LFHF TG}}
  {\phantomsubcaption\label{fig:LFHF ATG genes}}
   {\phantomsubcaption\label{fig:OPLA 0uM Picture}}
    {\phantomsubcaption\label{fig:OPLA 800uM Picture}}
  \tikz\node[inner sep=0pt,label={[anchor=north west]north west:\subref{fig:LFHF TG}}] {\includegraphics[width=0.49\textwidth]{figures/ch3-Model Development/LFHF TG.png}};
  \tikz\node[inner sep=0pt,label={[anchor=north west]north west:\subref{fig:LFHF ATG genes}}] {\includegraphics[width=0.49\textwidth]{figures/ch3-Model Development/LFHF ATG genes.png}};
   \tikz\node[inner sep=0pt,label={[anchor=north west]north west:\subref{fig:OPLA 0uM Picture}}] {\includegraphics[width=0.49\textwidth]{figures/ch3-Model Development/OPLA 0uM Picture.png}};
      \tikz\node[inner sep=0pt,label={[anchor=north west]north west:\subref{fig:OPLA 800uM Picture}}] {\includegraphics[width=0.49\textwidth]{figures/ch3-Model Development/OPLA 800uM Picture.png}};
    \caption{\textbf{Effect of media FA concentration on TG accumulation and autophagy.} Huh7 cells were cultured for 7 days in increasing concentrations of OPLA fatty acid mix. \textbf{A)} Lipids were extracted and intracellular TG content quantified by gas chromatography. \textbf{B)} RNA was extracted to measure autophagic gene expression relative to three housekeeper genes. Representative images (40x magnification) of cells cultured with \textbf{C)} no FA and in \textbf{D)} 800μM OPLA FA mix. Graphs representative of three biological repeats (n=3) carried out in technical triplicate. * p < 0.05. Abbreviations: TG, Triglyceride.}
            \label{fig:ch3-Model Development LFHF}
\end{figure}


\subsection{Effect of media FA composition on TG accumulation and autophagy}

Culturing Huh7 cells in predominantly unsaturated (OPLA) or saturated (POLA) FAs resulted in a similar increase in intracellular TG content, when compared to control media free from FAs (Figure \ref{fig:OPLAPOLA TG}). Intracellular TG composition reflected that of the FAs the cells were cultured in (Figure \ref{fig:OPLAPOLA Lipid}). Autophagic flux was then assessed by measuring LC3-II protein intensity (relative to housekeeper protein $\alpha$-Tubulin) with and without the autophagic inhibitor BAF. As LC3-II is normally degraded during autophagy, it's accumulation during autophagic inhibition is reflective of overall autophagic flux \cite{DJ2021Guidelines1}. There was no significant difference in autophagic flux between control, OPLA and POLA conditions, though there was a trend towards a decrease in cells treated with FAs (Figures \ref{fig:OPLAPOLA ATG FLX} - \ref{fig:OPLAPOLA WB Photo}).

\begin{figure}[h!]
  \centering
  {\phantomsubcaption\label{fig:OPLAPOLA TG}}
  {\phantomsubcaption\label{fig:OPLAPOLA Lipid}}
   {\phantomsubcaption\label{fig:OPLAPOLA ATG FLX}}
    {\phantomsubcaption\label{fig:OPLAPOLA BAF}}
     {\phantomsubcaption\label{fig:OPLAPOLA WB Photo}}
  \tikz\node[inner sep=0pt,label={[anchor=north west]north west:\subref{fig:OPLAPOLA TG}}] {\includegraphics[width=0.49\textwidth]{figures/ch3-Model Development/OPLAPOLA TG.png}};
  \tikz\node[inner sep=0pt,label={[anchor=north west]north west:\subref{fig:OPLAPOLA Lipid}}] {\includegraphics[width=0.49\textwidth]{figures/ch3-Model Development/OPLAPOLA Lipid contributions.png}};
   \tikz\node[inner sep=0pt,label={[anchor=north west]north west:\subref{fig:OPLAPOLA ATG FLX}}] {\includegraphics[width=0.49\textwidth]{figures/ch3-Model Development/OPLAPOLA ATG FLX.png}};
      \tikz\node[inner sep=0pt,label={[anchor=north west]north west:\subref{fig:OPLAPOLA BAF}}] {\includegraphics[width=0.49\textwidth]{figures/ch3-Model Development/OPLAPOLA BAF.png}};
      \tikz\node[inner sep=0pt,label={[anchor=north west]north west:\subref{fig:OPLAPOLA WB Photo}}] {\includegraphics[width=0.6\textwidth]{figures/ch3-Model Development/OPLAPOLA WB Photo.jpg}};
        \caption{\textbf{Effect of media FA composition on TG accumulation and autophagy.} Cells were cultured for 7 days in either media with no FA (Control), a predominantly unsaturated FA mix (OPLA) or a predominantly saturated FA mix (POLA). Lipids were extracted and quantified by gas chromatography to measure \textbf{A)} total intracellular TG and \textbf{B)} the composition of intracellular TG. \textbf{C-E)} Autophagic flux was calculated as LC3-II intensity (relative to housekeeper protein $\alpha$-Tubulin): (BAF - Basal)/Basal. Representative of three biological repeats (n=3) carried out in technical duplicate. * p < 0.05, ** p < 0.01, *** p < 0.001. Abbreviations: BAF, Bafilomycin A1; TG, Triglyceride.}
        \label{fig:OPLAPOLA}
\end{figure}