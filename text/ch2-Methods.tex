\chapter{\label{ch:2-Methods}Methods}

\minitoc

\section{Cell culture procedures}

\subsection{Routine cell culture}

Human hepatocellular carcinoma cells (Huh7) were kindly provided by Dr Camilla Pramfalk (Karolinska Institutet) and human hepatic stellate cells (LX2) by Hamish Miller (Blizard Institute). Cells were cultured in maintenance media made from Dulbecco's modified Eagle's medium (DMEM) supplemented with GlutaMAX\textsuperscript{TM}, 5.5 mmol/L glucose, 1 mM sodium pyruvate, 2.4 g/L sodium bicarbonate, 10\% foetal bovine serum (FBS), 10,000 U/ml penicillin-streptomycin (P/S) and 1\% non-essential amino acids (NEAA). Maintenance media was changed every 2-3 days and cells kept in a incubator at 37\textdegree C and 5\% carbon dioxide. 

\subsection{Passaging cells}

Cells were grown to at least 80\% confluence in T175 (175 cm\textsuperscript{2}) flasks before being washed with phosphate buffered saline (PBS) and then incubated for 5 mins in 4-5 mL of TrypLE\textsuperscript{TM}. Flasks were then agitated to dislodge cells and 8 mL of maintenance media was added. Cells were centrifuged for at 800 rpm for 3 mins at 21\textdegree C. The supernatant was removed and the pellet re-suspended in 5 mL of maintenance media before being seeded 1:5 in new T175 flasks with maintenance media.

\subsection{Seeding cells for experiments}

Cells were isolated and re-suspended in maintenance media as described for passaging. 20 \uL of cells were added to both sides of a cell counter slide and cells were counted using a Cellometer Auto T4 Bright Field Cell Counter. An average of both sides was taken and cells were then diluted with maintenance media to a concentration of 1 x 10\textsuperscript{5} cells per mL. Total cell number per well were as follows: 6-well plate - 2 x 10\textsuperscript{5}, 12-well plate - 1 x 10\textsuperscript{5}, 24-well plate - 0.5 x 10\textsuperscript{5}, 96-well plate - 0.1 x 10\textsuperscript{5}.

\subsection{Freezing cells for storage in liquid nitrogen}

During passaging, cells were isolated but resuspended in 10 mL freezing media made from 90\% maintenance media and 10\% dimethyl sulfoxide (DMSO). 1 mL of the cell suspension was placed into each cryovial before being stored overnight in a -80\textdegree C freezer in a Styrofoam holder to slow the rate of cooling. Cells were then transferred to liquid nitrogen for long-term storage.  

\subsection{Thawing cells from storage in liquid nitrogen}

Frozen cells were placed in a 37\textdegree C water bath to thaw rapidly. 500 \uL of warm maintenance media was then added to the cryovial and mixed with a pipette. Cells were then seeded in a T75 (75cm\textsuperscript{2}) flask with maintenance media. Media was changed the next day and then every other day until cells reached at least 80\% and were passaged into a T175 flask.

\section {Quantitative reverse-transcription PCR}

\subsection{RNA extraction}

Cells for were cultured in 12-well plates and upon harvesting were washed twice with warm PBS before 250 \uL of TRI Reagent\textsuperscript{\textregistered}  was added to each well. Cells were dislodged from the plate using a cell scraper and transferred into Eppendorf tubes to be frozen at -80\textdegree C until required. When required, samples were added to Phase Lock Gel tubes with 50 \uL of chloroform and shaken vigorously. Samples were left for 5 mins and then centrifuged for 15 mins at 1200 g and 4\textdegree C. The clear supernatant was transferred to 2 mL tapered-bottom elution tubes containing 2 \uL glycol blue and 200 \uL isopropanol before being left overnight at -20\textdegree C. 

Samples were then centrifuged at 1200 g and 4\textdegree C for 30 mins to allow a RNA pellet to form. The supernatant was removed and pellet resuspended in 80\% ethanol before being centrifuged again at 1200 g and 4\textdegree C for 15 mins. The ethanol was then removed and the pellet allowed to dry before re-suspension in 15 \uL RNAse free water and incubation for 3 hours. RNA purity and concentration was then measured using a NanoDrop ND-1000 Spectrophotometer.

\subsection{Reverse transcription}

The volume of sample containing 1 \ug RNA was transferred into a 96 well plate and made up to 10 \uL with RNAse-free water. For each well with sample, 2 \uL 10x reverse transcription buffer, 0.8 \uL 25x deoxynucleoside triphosphates (dNTPs), 2 \uL 10x random primers, 1 \uL reverse transcriptase and 4.2 \uL RNAse-free water was added. The plate was sealed and briefly spun before being placed on an MJ Research Tetrad PTC-225 Thermal Cycler. The thermal cycler heated the plate to 27 \textdegree C for 10 minutes for the random primers to anneal to the sample RNA and this was then followed by 120 minutes at 37\textdegree C for the reverse transcriptase to form the cDNA using the dNTPs. Finally the reaction was terminated by heating the plate to 85\textdegree C for 5 mins. The cDNA could then be stored in the freezer until required. 

\subsection{Quantative PCR}

The samples were diluted 1:40 in 10 mmol/L Trizma\R hydrochloride and measured in triplicate for each gene. A standard curve was made through serial dilution of cDNA pooled from all of the samples in 10 mmol/L Trizma\R and included for each gene. Quantative PCR was carried out on \textbf{Insert name of machine} and TaqMan\R assays used to measure gene expression. A 6 \uL reaction volume was used consisting of 2.7 \uL of sample cDNA, 0.3 \uL TaqMan\R probes and 3 \uL 2X KAPA PROBE FAST Master Mix. Relative expression was calculated as a ratio from generated Ct values as previously described \cite{Pfaffl2001ARTPCR}. Each sample was expressed relative to a calibrator sample and corrected to the geometric mean of three reference housekeeper genes (B2M, HMBS, and YWHAZ).