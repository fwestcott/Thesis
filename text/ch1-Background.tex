\chapter{\label{ch:1-background}Background} 

\minitoc

\section{Obesity}

In 2016 the World Health Organisation estimated 1.9 billion adults worldwide were classified as overweight, with 650 million of these classified as obese  \cite{WHO2016factsheetObesityOverweight}, and the prevalence has continued to increase rapidly \cite{WHO2016factsheetObesityOverweight, Finucane2011NationalParticipants}. Obesity is a major risk factor for insulin resistance (IR), type-2 diabetes mellitus (T2D) and cardiovascular disease (CVD) \cite{Reaven1995PathophysiologyDisease, DeFronzo2015TypeMellitus, Khan2018AssociationMorbidity}; it is strongly associated with fat deposition in non-adipose tissue organs (ectopic fat), including the liver, heart, and skeletal muscle.

\section{The liver}

\subsection{Structure}

Anatomically the liver can be split into four main lobes that consist of small hexagonal lobules containing the four cell types: 1) sinusoidal endothelial cells, 2) stellate cells, 3) Kupffer cells, and 4) hepatocytes. Along each corner of the lobule there is the portal triad made up of the hepatic artery, the hepatic portal vein, and the common bile duct. As well as the oxygen-rich blood from the hepatic artery, the liver is unique in that it also receives nutrient-rich deoxygenated blood directly from the pancreas, spleen and the entire gastrointestinal tract through the hepatic portal vein. The blood from both vessels combine in the liver sinusoid and travel down to a central vein in the middle of each lobule which will eventually drain into the hepatic vein and then the inferior vena cava. The liver sinusoid is lined with sinusoidal endothelial cells and liver-specific tissue-resident macrophages known as Kuppfer cells which remove debris and foreign particles through phagocytosis. The endothelial cells are highly fenestrated, which allows communication between the sinusoid and the hepatocytes and stellate cells.

\subsection{Function}

The liver has many hundreds of functions, including the production of bile, regulation of systemic iron levels, synthesis of amino acids and cholesterol, regulation of blood clotting and detoxification of the blood. One of the main functions is the regulation and metabolism of key nutritional substrates in the blood such as glucose and fatty acids. These processes are primarily under the control of pancreatic hormones insulin and glucagon which the liver gets the first pass of and are themselves regulated by nutritional state.  

\subsubsection{Glucose Metabolism}

In the fasting state blood glucose is maintained by endogenous glucose production which occurs almost exclusively in the liver. Glycogen stores are broken down into glucose through glycogenolysis (GLY) and new glucose molecules are made from non-carbohydrate precursors (such as amino acids and glycerol) through gluconeogenesis (GNG). An increase in glucagon and decrease in insulin levels in the blood is responsible for this up-regulation of GNG and GLY. In the transition to the postprandial state after the consumption of a mixed meal, glucagon secretion is inhibited and plasma insulin concentrations sharply rise. This shifts hepatic metabolism to a decrease in glucose production and an increase in glucose uptake, storage and oxidation. 

\subsubsection{Fatty Acid Metabolism}


Triglyceride (TG) is a crucial metabolic substrate for many parts of the body. TG can be synthesised in the liver from non-lipid precursors through a process called \textit{de novo} lipogeneis (DNL) where it is then stored in lipid droplets(LD) or secreted packaged in TG-enriched very low-density lipoprotein (VLDL-TG). DNL is up-regulated by the presence of excess carbohydrates as well as the hormone insulin. 

\section{Non-alcoholic fatty liver disease}

\section{Autophagy}